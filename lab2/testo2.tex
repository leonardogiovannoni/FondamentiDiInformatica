\documentclass[a4paper,12pt]{article}

\usepackage[utf8]{inputenc}
\usepackage[T1]{fontenc}
\usepackage{listings}
\usepackage{xcolor}
\usepackage{geometry}
\geometry{margin=2.5cm}

% Stile per il codice C++
\lstset{
    language=C++,
    basicstyle=\ttfamily\small,
    keywordstyle=\color{blue},
    commentstyle=\color{gray},
    stringstyle=\color{orange},
    numbers=left,
    numberstyle=\tiny\color{gray},
    stepnumber=1,
    numbersep=10pt,
    tabsize=4,
    showspaces=false,
    showstringspaces=false,
    frame=single,
    breaklines=true,
    breakatwhitespace=true,
    captionpos=b
}

\title{Esercizio 2: Stampa triangolo di asterischi}
\author{}
\date{}

\begin{document}

\maketitle
\thispagestyle{empty}
\pagestyle{empty}

\section*{Testo dell'esercizio}

Scrivi un programma in linguaggio \textbf{C++} che:

\begin{enumerate}
    \item \textbf{Legga} da tastiera un numero intero positivo \texttt{n}.
    \item \textbf{Stampi} a video un \textbf{triangolo rettangolo di asterischi} con \texttt{n} righe, dove:
    \begin{itemize}
        \item la prima riga contiene 1 asterisco (\texttt{*}),
        \item la seconda riga contiene 2 asterischi,
        \item la terza riga 3 asterischi,
        \item e così via, fino alla riga \texttt{n}, che contiene \texttt{n} asterischi.
    \end{itemize}
\end{enumerate}

\section*{Esempio}

\noindent\textbf{Input:}
\begin{verbatim}
4
\end{verbatim}

\noindent\textbf{Output:}
\begin{verbatim}
*
**
***
****
\end{verbatim}

\end{document}

