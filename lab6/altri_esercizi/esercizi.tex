\documentclass[a4paper,12pt]{article}
\usepackage[utf8]{inputenc}
\usepackage[T1]{fontenc}
\usepackage{amsmath,amssymb}
\usepackage{enumitem}
\usepackage{geometry}
\usepackage{hyperref}
\usepackage{minted}
\usepackage{textcomp}
\usepackage{xparse}

%\ExplSyntaxOn
%\RenewDocumentCommand{\texttt}{m}
% {
%  \tl_set:Nn \l_nemgathos_upquotes_tl { #1 }
%  \tl_replace_all:Nnn \l_nemgathos_upquotes_tl { '' } { \textquotedbl }
%  \tl_replace_all:Nnn \l_nemgathos_upquotes_tl { `` } { \textquotedbl }
%  \tl_replace_all:Nnn \l_nemgathos_upquotes_tl { ' } { \textquotesingle }
%  \tl_replace_all:Nnn \l_nemgathos_upquotes_tl { ` } { \textquotesingle }
%  { \ttfamily \tl_use:N \l_nemgathos_upquotes_tl }
% }
%\tl_new:N \l_nemgathos_upquotes_tl
%\ExplSyntaxOff
\geometry{margin=2.5cm}
\setlength{\parindent}{0pt}

\title{Esercizi di Programmazione (C++): Array, Cstring e Matrici}
\author{}
% insert no date
\date{}

\begin{document}
\maketitle
\setlist[itemize]{noitemsep, topsep=4pt}

%%%%%%%%%%%%%%%%%%%%%%%%%%%%%%%%%%%%%%%%%%%%%%%%%%%%%%%%%%%%%
\section{Implementazione della funzione \texttt{my\_strlen}}
\textbf{Testo.}  
Scrivi una funzione:
\begin{minted}{cpp}
int my_strlen(const char s[]);
\end{minted}
che calcola la \textbf{lunghezza} di una stringa C-style senza usare la libreria standard.

Nel \texttt{main}:
\begin{itemize}
    \item dichiara un array \texttt{str} di capacità 51;
    \item leggi una stringa da input;
    \item calcola la lunghezza usando \texttt{my\_strlen};
    \item stampa la lunghezza calcolata.
\end{itemize}

\subsection*{Esempi di esecuzione.}

\noindent
\begin{minipage}[t]{0.48\textwidth}
\textbf{Esempio 1}\\
\textbf{Input:}
\begin{verbatim}
ciao
\end{verbatim}
\textbf{Output:}
\begin{verbatim}
Lunghezza: 4
\end{verbatim}
\end{minipage}
\hfill
\begin{minipage}[t]{0.48\textwidth}
\textbf{Esempio 2}\\
\textbf{Input:}
\begin{verbatim}
programmazione
\end{verbatim}
\textbf{Output:}
\begin{verbatim}
Lunghezza: 14
\end{verbatim}
\end{minipage}

%%%%%%%%%%%%%%%%%%%%%%%%%%%%%%%%%%%%%%%%%%%%%%%%%%%%%%%%%%%%%
\section{Copia dei soli caratteri alfabetici}
\textbf{Testo.}  
Scrivi una funzione:
\begin{minted}{cpp}
void filtra_alfabetici(char dst[], int dst_capacity, const char src[]);
\end{minted}
che copia in \texttt{dst} soltanto i caratteri alfabetici (\texttt{A--Z}, \texttt{a--z}) contenuti in \texttt{src}, rispettando la capacità \texttt{capacity}.

Nel \texttt{main}:
\begin{itemize}
    \item leggi una stringa in \texttt{str1} (capacità 10),
    \item dichiara \texttt{str2} (capacità 100),
    \item chiama \texttt{filtra\_alfabetici(str2, 10, str1)},
    \item stampa il risultato.
\end{itemize}

\subsection*{Esempi di esecuzione.}

\noindent
\begin{minipage}[t]{0.48\textwidth}
\textbf{Esempio 1}\\
\textbf{Input:}
\begin{verbatim}
C++17_e'_potente!
\end{verbatim}
\textbf{Output:}
\begin{verbatim}
Cepotente
\end{verbatim}
\end{minipage}
\hfill
\begin{minipage}[t]{0.48\textwidth}
\textbf{Esempio 2}\\
\textbf{Input:}
\begin{verbatim}
Hello123World!
\end{verbatim}
\textbf{Output:}
\begin{verbatim}
HelloWorl
\end{verbatim}
\end{minipage}
\newline \newline
\textit{Nota: la capacità di \texttt{str2} è 10, quindi l'output contiene al massimo 9 caratteri più il terminatore nullo, per cui "HelloWorld" diventa "HelloWorl".}

%%%%%%%%%%%%%%%%%%%%%%%%%%%%%%%%%%%%%%%%%%%%%%%%%%%%%%%%%%%%%
\section{Somma degli elementi per riga in una matrice lineare}
\textbf{Testo.}  
Implementa una funzione:
\begin{minted}{cpp}
int somma_riga(const int *m, int cols, int rows, int r);
\end{minted}
che somma gli elementi della riga indicata.
Nel \texttt{main}:
\begin{itemize}
    \item leggi una matrice $4 \times 5$ in una matrice;
    \item per ogni riga, stampa la relativa somma.
\end{itemize}

\subsection*{Esempio di esecuzione.}

\noindent
\textbf{Input:}
\begin{verbatim}
1 2 3 4 5
6 7 8 9 10
11 12 13 14 15
16 17 18 19 20
\end{verbatim}

\textbf{Output:}
\begin{verbatim}
Somma riga 0: 15
Somma riga 1: 40
Somma riga 2: 65
Somma riga 3: 90
\end{verbatim}

%%%%%%%%%%%%%%%%%%%%%%%%%%%%%%%%%%%%%%%%%%%%%%%%%%%%%%%%%%%%%
\section{Trasposizione di una matrice rettangolare}
\textbf{Testo.}  
Scrivi una funzione:
\begin{minted}{cpp}
void trasponi(const int *src, int *dst, int rows, int cols);
\end{minted}

che memorizza in \texttt{dst} la matrice trasposta della matrice \texttt{src}.

La trasposizione soddisfa:

% scrivi la formula matematica compatta della trasposta in latex

\[
    \left(A^T\right)_{ij} = A_{ji} \qquad \forall A \in \mathbb{Z}^{m,n} \quad 1 \le i \le m, \quad 1 \le j \le n
\]

\[
\begin{bmatrix}
a_{11} & a_{12} & \cdots & a_{1n} \\
a_{21} & a_{22} & \cdots & a_{2n} \\
\vdots & \vdots & \ddots & \vdots \\
a_{m1} & a_{m2} & \cdots & a_{mn}
\end{bmatrix}^T =
\begin{bmatrix}
a_{11} & a_{21} & \cdots & a_{m1} \\
a_{12} & a_{22} & \cdots & a_{m2} \\
\vdots & \vdots & \ddots & \vdots \\
a_{1n} & a_{2n} & \cdots & a_{mn}
\end{bmatrix}
\]

Nel \texttt{main}:
\begin{itemize}
    \item leggi una matrice $3 \times 4$;
    \item trasponila in una matrice $4 \times 3$;
    \item stampa il risultato.
\end{itemize}

\subsection*{Esempio di esecuzione.}

\noindent
\textbf{Input:}
\begin{verbatim}
1 2 3 4
5 6 7 8
9 10 11 12
\end{verbatim}

\textbf{Output:}
\begin{verbatim}
1 5 9
2 6 10
3 7 11
4 8 12
\end{verbatim}

%%%%%%%%%%%%%%%%%%%%%%%%%%%%%%%%%%%%%%%%%%%%%%%%%%%%%%%%%%%%%
\section{Verifica dell'alfabeto italiano (21 lettere)}
\textbf{Testo.}  
Scrivi una funzione:
\begin{minted}{cpp}
bool is_alfabeto_italiano_21(const char s[]);
\end{minted}
che restituisce \texttt{true} solo se la stringa è composta esclusivamente dalle \textbf{21 lettere dell'alfabeto italiano}, maiuscole o minuscole.

Nel \texttt{main}:
\begin{itemize}
    \item leggi una stringa di lunghezza massima 50;
    \item verifica che rispetti l'alfabeto italiano;
    \item stampa \texttt{OK} oppure \texttt{NO}.
\end{itemize}

\subsection*{Esempi di esecuzione.}

\noindent
\begin{minipage}[t]{0.48\textwidth}
\textbf{Esempio 1}\\
\textbf{Input:}
\begin{verbatim}
casa
\end{verbatim}
\textbf{Output:}
\begin{verbatim}
OK
\end{verbatim}
\end{minipage}
\hfill
\begin{minipage}[t]{0.48\textwidth}
\textbf{Esempio 2}\\
\textbf{Input:}
\begin{verbatim}
yogurt
\end{verbatim}
\textbf{Output:}
\begin{verbatim}
NO
\end{verbatim}
\end{minipage}

\vspace{0.3cm}
\noindent

\end{document}

