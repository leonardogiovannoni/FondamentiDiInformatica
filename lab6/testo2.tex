\documentclass[a4paper,11pt]{article}

\usepackage[utf8]{inputenc}
\usepackage[T1]{fontenc}
\usepackage{listings}
\usepackage{xcolor}
\usepackage{amsmath}
\usepackage{amssymb}
\usepackage{geometry}
\usepackage{minted}
\usepackage{multicol}
\geometry{margin=1cm}

% Stile per il codice C++
\lstset{
    language=C++,
    basicstyle=\ttfamily\small,
    keywordstyle=\color{blue},
    commentstyle=\color{gray},
    stringstyle=\color{orange},
    numbers=left,
    numberstyle=\tiny\color{gray},
    stepnumber=1,
    numbersep=10pt,
    tabsize=4,
    showspaces=false,
    showstringspaces=false,
    frame=single,
    breaklines=true,
    breakatwhitespace=true,
    captionpos=b
}

\title{Esercizio: Lettura, stampa e manipolazione di matrici}
\author{}
\date{}

\begin{document}

\maketitle
\thispagestyle{empty}
\pagestyle{empty}

\section*{Parte A: Lettura e stampa di una matrice}

Scrivi un programma in linguaggio \textbf{C++} che:

\begin{itemize}
    \item dichiari una matrice di dimensione fissa \( 2 \times 3 \);
    \item legga tutti gli elementi della matrice da tastiera, riempiendola riga per riga;
    \item stampi la matrice a video, mantenendo la struttura a righe e colonne.
\end{itemize}

Le funzioni da utilizzare saranno le seguenti:

\begin{minted}{cpp}
void leggi_matrice(int *m, int rows, int cols);
void stampa_matrice(int *m, int rows, int cols);
\end{minted}

\section*{Esempio}


\noindent\textbf{Input:}
\begin{minted}{text}
1 2 3 4 5 6
\end{minted}

\noindent\textbf{Output:}
\begin{minted}{text}
1 2 3
4 5 6
\end{minted}

\vspace{0em}

\section*{Parte B: Inversione delle colonne di una matrice}

Estendi il programma precedente aggiungendo una funzione che \textbf{inverta l'ordine delle colonne} della matrice, per ogni riga.

\begin{itemize}
    \item Dopo aver letto e stampato la matrice, il programma deve invertire le colonne della matrice.
    \item Successivamente deve stamparla nuovamente.
\end{itemize}

La funzione da utilizzare sarà:

\begin{minted}{cpp}
void inverti_colonne(int *m, int rows, int cols);
\end{minted}

\section*{Esempio}

\noindent\textbf{Input:}
\begin{minted}{text}
1 2 3 4 5 6
\end{minted}

\noindent\textbf{Output:}
\begin{minted}{text}
1 2 3
4 5 6
Matrice dopo inversione colonne:
3 2 1
6 5 4
\end{minted}

\end{document}

