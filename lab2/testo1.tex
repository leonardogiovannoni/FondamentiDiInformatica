\documentclass[a4paper,12pt]{article}

\usepackage[utf8]{inputenc}
\usepackage[T1]{fontenc}
\usepackage{listings}
\usepackage{xcolor}
\usepackage{geometry}
\geometry{margin=2.5cm}

% Stile per il codice C++
\lstset{
    language=C++,
    basicstyle=\ttfamily\small,
    keywordstyle=\color{blue},
    commentstyle=\color{gray},
    stringstyle=\color{orange},
    numbers=left,
    numberstyle=\tiny\color{gray},
    stepnumber=1,
    numbersep=10pt,
    tabsize=4,
    showspaces=false,
    showstringspaces=false,
    frame=single,
    breaklines=true,
    breakatwhitespace=true,
    captionpos=b
}

\title{Esercizio 1: Stampa numeri da 0 a n-1}
\author{}
\date{}

\begin{document}

\maketitle
\thispagestyle{empty}
\pagestyle{empty}

\section*{Testo dell'esercizio}

Scrivi un programma in linguaggio \textbf{C++} che:

\begin{enumerate}
    \item \textbf{Legga} da tastiera un numero intero positivo \texttt{n}.
    \item \textbf{Stampi} tutti i numeri interi da \texttt{0} a \texttt{n - 1}, ognuno su una riga.
\end{enumerate}
\section*{Esempio}

\noindent\textbf{Input:}
\begin{verbatim}
5
\end{verbatim}

\noindent\textbf{Output:}
\begin{verbatim}
0
1
2
3
4
\end{verbatim}
\end{document}

