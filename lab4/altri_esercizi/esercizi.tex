\documentclass[a4paper,12pt]{article}
\usepackage[utf8]{inputenc}
\usepackage[T1]{fontenc}
\usepackage{amsmath,amssymb}
\usepackage{enumitem}
\usepackage{geometry}
\usepackage{hyperref}
\usepackage{minted}
\geometry{margin=2.5cm}
\setlength{\parindent}{0pt}

\title{Esercizi di Programmazione (C++)}
\author{}
\date{}

\begin{document}
\maketitle
\setlist[itemize]{noitemsep, topsep=4pt}

\section{Somma degli elementi di un array}
\textbf{Testo.} Leggi 5 numeri interi e memorizzali in un array. Calcola e stampa la \textbf{somma} di tutti gli elementi.
\newline 
\newline
\textbf{Esempio di esecuzione.}
\newline 
\newline
\noindent\textbf{Input:}
\begin{verbatim}
5 7 2 -1 4
\end{verbatim}

\noindent\textbf{Output:}
\begin{verbatim}
Somma = 17
\end{verbatim}

\section{Valore massimo e minimo}
\textbf{Testo.} Leggi 6 numeri interi in un array. Trova e stampa il \textbf{valore massimo} e il \textbf{valore minimo} tra essi.
\newline 
\newline
\textbf{Esempio di esecuzione.}
\newline 
\newline
\noindent\textbf{Input:}
\begin{verbatim}
4 9 -2 7 1 5
\end{verbatim}

\noindent\textbf{Output:}
\begin{verbatim}
Massimo = 9
Minimo = -2
\end{verbatim}

\section{Conta i numeri negativi}
\textbf{Testo.} Leggi 10 numeri interi e conta quanti di essi sono \textbf{negativi}. Stampa il conteggio finale.
\newline 
\newline
\textbf{Esempio di esecuzione.}
\newline 
\newline
\noindent\textbf{Input:}
\begin{verbatim}
5 -3 0 -7 12 9 -1 -4 6 8
\end{verbatim}

\noindent\textbf{Output:}
\begin{verbatim}
Numeri negativi: 4
\end{verbatim}

\section{Scambio con puntatori}
\textbf{Testo.} Scrivi una funzione che scambia due valori interi utilizzando i \textbf{puntatori}.  
La funzione deve avere la seguente firma:
\begin{minted}{cpp}
void scambia(int *a, int *b);
\end{minted}
Nel programma principale, leggi due interi, scambiali usando la funzione e stampa i nuovi valori.
\newline 
\newline
\textbf{Esempio di esecuzione.}
\newline 
\newline
\noindent\textbf{Input:}
\begin{verbatim}
3 10
\end{verbatim}

\noindent\textbf{Output:}
\begin{verbatim}
Dopo scambio: a = 10 b = 3
\end{verbatim}

\section{Media intera degli elementi}
\textbf{Testo.} Scrivi un programma che calcola la \textbf{media intera} dei valori contenuti in un array di numeri interi.  
\newline 
\newline
\textbf{Esempio di esecuzione.}
\newline 
\newline
\noindent\textbf{Input:}
\begin{verbatim}
4 6 8 2 10
\end{verbatim}

\noindent\textbf{Output:}
\begin{verbatim}
Media intera = 6
\end{verbatim}

\section{Verificare se un array è palindromo}
\textbf{Testo.} Leggi un array di numeri interi e verifica se è \textbf{palindromo}, cioè se il primo e l'ultimo elemento coincidono, il secondo e il penultimo, e così via. 
Più rigorosamente, un vettore $v = (v_0, v_1, \dots, v_{n-1})$ è \textbf{palindromo} se e solo se
\[
v_i = v_{n - 1 - i} \quad \forall \, i \in \{0, 1, 2, \dots, \lfloor \tfrac{n}{2} \rfloor - 1\}.
\]

Se la condizione è vera per tutte le coppie, stampa \texttt{si}, altrimenti \texttt{no}.
\newline 
\newline
\textbf{Esempio di esecuzione.}
\newline 
\newline
\noindent\textbf{Input:}
\begin{verbatim}
1 2 3 2 1
\end{verbatim}

\noindent\textbf{Output:}
\begin{verbatim}
palindromo? si
\end{verbatim}

\end{document}

