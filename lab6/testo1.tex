\documentclass[a4paper,10pt]{article}

\usepackage[utf8]{inputenc}
\usepackage[T1]{fontenc}
\usepackage{listings}
\usepackage{xcolor}
\usepackage{amsmath}
\usepackage{amssymb}
\usepackage{geometry}
\usepackage{minted}
\usepackage{multicol}
\usepackage{textcomp}
\usepackage{xparse}

\ExplSyntaxOn
\RenewDocumentCommand{\texttt}{m}
 {
  \tl_set:Nn \l_nemgathos_upquotes_tl { #1 }
  \tl_replace_all:Nnn \l_nemgathos_upquotes_tl { '' } { \textquotedbl }
  \tl_replace_all:Nnn \l_nemgathos_upquotes_tl { `` } { \textquotedbl }
  \tl_replace_all:Nnn \l_nemgathos_upquotes_tl { ' } { \textquotesingle }
  \tl_replace_all:Nnn \l_nemgathos_upquotes_tl { ` } { \textquotesingle }
  { \ttfamily \tl_use:N \l_nemgathos_upquotes_tl }
 }
\tl_new:N \l_nemgathos_upquotes_tl
\ExplSyntaxOff
\geometry{margin=0.5cm}

% Stile per il codice C++
\lstset{
    language=C++,
    basicstyle=\ttfamily\small,
    keywordstyle=\color{blue},
    commentstyle=\color{gray},
    stringstyle=\color{orange},
    numbers=left,
    numberstyle=\tiny\color{gray},
    stepnumber=1,
    numbersep=10pt,
    tabsize=4,
    showspaces=false,
    showstringspaces=false,
    frame=single,
    breaklines=true,
    breakatwhitespace=true,
    captionpos=b
}

\title{Esercizio: Manipolazione di stringhe in C++}
\author{}
\date{}

\begin{document}

\maketitle
\thispagestyle{empty}
\pagestyle{empty}

% ============================================================
\section*{Parte A: Implementazione \texttt{my\_strcpy}}

Scrivi un programma in linguaggio \textbf{C++} che:

\begin{itemize}
    \item dichiari due array di caratteri di lunghezza prefissata 51;
    \item legga da tastiera una stringa e la memorizzi nel primo array;
    \item visualizzi la stringa;
    \item utilizzi una funzione \texttt{my\_strcpy} per copiare il contenuto della stringa nel secondo array;
    \item stampi la stringa contenuta nel secondo array sia prima che dopo la copia.
\end{itemize}

La funzione da implementare è la seguente, dove il valore ritornato è \texttt{dst}:

\begin{minted}{cpp}
char *my_strcpy(char dst[], const char src[]);
\end{minted}

\section*{Esempio}

%\begin{multicols}{2}
\noindent\textbf{Input:}
\begin{minted}{text}
abc
\end{minted}

\noindent\textbf{Output:}
\begin{minted}{text}
str1: abc 
str2 PRIMA: 
str2 DOPO: abc 
\end{minted}
%\end{multicols}

\vspace{2em}

% ============================================================
\section*{Parte B: Verifica di stringa composta da solo lettere maiuscole o cifre}

Estendi il programma precedente introducendo una funzione che verifichi se una stringa è composta esclusivamente da:

\begin{itemize}
    \item lettere dell'alfabeto inglese in \textbf{maiuscolo} (da \texttt{'A'} a \texttt{'Z'});
    \item cifre numeriche decimali (da \texttt{'0'} a \texttt{'9'}).
\end{itemize}
La funzione deve restituire \texttt{true} se tutti i caratteri della stringa rispettano tali condizioni, \texttt{false} altrimenti.
La funzione da implementare è:

\begin{minted}{cpp}
bool is_alpha_maiusc_or_numeric(const char str[]);
\end{minted}

\section*{Esempio}

\begin{multicols}{2}
\noindent\textbf{Input:}
\begin{minted}{text}
ABC123
\end{minted}

\noindent\textbf{Output:}
\begin{minted}{text}
str1: ABC123
str2 PRIMA: 
str2 DOPO: ABC123
OK
\end{minted}

\columnbreak

\noindent\textbf{Input:}
\begin{minted}{text}
AbC!12
\end{minted}

\noindent\textbf{Output:}
\begin{minted}{text}
str1: AbC!12 
str2 PRIMA: 
str2 DOPO: AbC!12
NO
\end{minted}
\end{multicols}

\end{document}

