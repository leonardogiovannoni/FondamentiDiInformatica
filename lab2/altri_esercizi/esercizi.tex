\documentclass[a4paper,12pt]{article}
\usepackage[utf8]{inputenc}
\usepackage[T1]{fontenc}
\usepackage{amsmath,amssymb}
\usepackage{enumitem}
\usepackage{geometry}
\usepackage{hyperref}
\geometry{margin=2.5cm}
\setlength{\parindent}{0pt}

\title{Esercizi di Programmazione (C++)}
\author{}
\date{}

\begin{document}
\maketitle
\setlist[itemize]{noitemsep, topsep=4pt}


\section{Somma dei primi $n$ numeri naturali a partire da 0}
\textbf{Testo.} Leggi un intero $n$ e calcola la somma dei primi $n$ numeri naturali a partire da $0$, cioè $0 + 1 + 2 + \dots + (n-1)$. Stampa la somma. Se $n \le 0$, la somma vale $0$.
\newline 
\newline
\textbf{Esempio di esecuzione.}
\begin{verbatim}
Input
5

Output
La somma fa: 10
\end{verbatim}

\section{Fattoriale di $n$}
\textbf{Testo.} Leggi un intero $n \ge 0$ e calcola il \textbf{fattoriale} $n! = 1 \cdot 2 \cdot \ldots \cdot n$.
\newline 
\newline 
\textbf{Esempio di esecuzione.}
\begin{verbatim}
Input
5

Output
Il fattoriale e' 120
\end{verbatim}

\section{Numeri pari in \texorpdfstring{$[0, n)$}{[0, n)} con conteggio e somma}
\textbf{Testo.} Leggi un intero $n$ e stampa tutti i numeri pari compresi tra $0$ e $n-1$. Al termine stampa quante occorrenze sono state stampate e la loro somma complessiva.
\newline 
\newline 
\textbf{Esempio di esecuzione.}
\begin{verbatim}
Input
10

Output
0
2
4
6
8
Pari stampati: 5
Somma dei pari: 20
\end{verbatim}

\section{Tabellina di un numero}
\textbf{Testo.} Leggi un intero $n$ e stampa la \textbf{tabellina di $n$} dalla moltiplicazione per $1$ alla moltiplicazione per $10$, nel formato \texttt{n x i = risultato}.
\newline 
\newline
\textbf{Esempio di esecuzione.}
\begin{verbatim}
Input
7

Output
7 x 1 = 7
7 x 2 = 14
7 x 3 = 21
7 x 4 = 28
7 x 5 = 35
7 x 6 = 42
7 x 7 = 49
7 x 8 = 56
7 x 9 = 63
7 x 10 = 70
\end{verbatim}

\section{Minimo, massimo e media intera di $N$ numeri}
\textbf{Testo.} Leggi un intero $N>0$ e poi leggi $N$ interi. Stampa il \textbf{minimo}, il \textbf{massimo} e la \textbf{media intera}.
\newline \newline \textbf{Esempio di esecuzione.}
\begin{verbatim}
Input
5
3 10 -2 7 7

Output
Minimo: -2
Massimo: 10
Media intera: 5
\end{verbatim}

\section{I primi $n$ numeri della sequenza di Fibonacci}
\textbf{Testo.} Leggi un intero $n$ e stampa i primi $n$ numeri della sequenza di Fibonacci, a partire da 0 e 1. Se $n \le 0$, non stampare nulla; se $n=1$, stampa solo 0.
\newline \newline \textbf{Esempio di esecuzione.}
\begin{verbatim}
Input
7

Output
0
1
1
2
3
5
8
\end{verbatim}

\section{Numero di cifre, somma delle cifre e numero invertito}
\textbf{Testo.} Leggi un intero non negativo $n$. Stampa: (1) quante cifre ha, (2) la somma delle cifre, (3) il numero ottenuto invertendo l'ordine delle cifre. Gestisci correttamente il caso $n=0$.
\newline \newline \textbf{Esempio di esecuzione.}
\begin{verbatim}
Input
12345

Output
Cifre: 5
Somma cifre: 15
Invertito: 54321
\end{verbatim}

\end{document}