\documentclass[a4paper,12pt]{article}
\usepackage[utf8]{inputenc}
\usepackage[T1]{fontenc}
\usepackage{amsmath,amssymb}
\usepackage{enumitem}
\usepackage{geometry}
\usepackage{hyperref}
\geometry{margin=2.5cm}
\setlength{\parindent}{0pt}

\title{Esercizi di Programmazione (C++) — Ricorsione}
\author{}
\date{}


\begin{document}
\maketitle
\setlist[itemize]{noitemsep, topsep=4pt}

\section{Calcolo del fattoriale}
\textbf{Testo.} Scrivi una funzione ricorsiva che calcola il \textbf{fattoriale} di un numero intero non negativo $n$.  
Per definizione:
\[
f(n) =
\begin{cases}
1 & \text{se } n = 0, \\
n \cdot f(n-1) & \text{se } n > 0.
\end{cases}
\]
Il programma deve leggere un numero intero positivo e stampare il suo fattoriale.

\subsection*{Esempio di esecuzione.}
\noindent\textbf{Input:}
\begin{verbatim}
5
\end{verbatim}

\noindent\textbf{Output:}
\begin{verbatim}
Il fattoriale di 5 e': 120
\end{verbatim}

\section{Conta dei numeri positivi}
\textbf{Testo.} Leggi 8 numeri interi in un array e scrivi una funzione ricorsiva che conta quanti elementi sono \textbf{strettamente positivi}.  


\subsection*{Esempio di esecuzione.}
\noindent\textbf{Input:}
\begin{verbatim}
3 -1 7 0 4 -5 2 8
\end{verbatim}

\noindent\textbf{Output:}
\begin{verbatim}
Ci sono 5 numeri positivi.
\end{verbatim}

\section{Massimo elemento di un array}
\textbf{Testo.} Scrivi una funzione ricorsiva che calcola il \textbf{valore massimo} all'interno di un array di interi positivi.  

\subsection*{Esempio di esecuzione.}
\noindent\textbf{Input:}
\begin{verbatim}
4 9 2 7 1 5
\end{verbatim}

\noindent\textbf{Output:}
\begin{verbatim}
Il valore massimo e': 9
\end{verbatim}

\section{Ricerca di un valore nell'array}
\textbf{Testo.} Scrivi una funzione ricorsiva che stabilisce se un dato valore $x$ è \textbf{presente} in un array di interi.  

\subsection*{Esempio di esecuzione.}
\noindent\textbf{Input:}
\begin{verbatim}
3 7 1 9 2 6 4
9
\end{verbatim}

\noindent\textbf{Output:}
\begin{verbatim}
9 e' presente nell'array.
\end{verbatim}

\section{Inversione di un array}
\textbf{Testo.} Scrivi un programma che inverte un array usando una funzione ricorsiva.  

\subsection*{Esempio di esecuzione.}
\noindent\textbf{Input:}
\begin{verbatim}
1 2 3 4 5 6 7 8
\end{verbatim}

\noindent\textbf{Output:}
\begin{verbatim}
Array invertito: 8 7 6 5 4 3 2 1
\end{verbatim}

\end{document}

