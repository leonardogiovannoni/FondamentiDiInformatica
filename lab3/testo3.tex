\documentclass[a4paper,12pt]{article}
\usepackage{amsmath}
\usepackage[utf8]{inputenc}
\usepackage[T1]{fontenc}
\usepackage{listings}
\usepackage{xcolor}
\usepackage{geometry}
\usepackage{multicol}
\geometry{margin=2.5cm}

% Stile per il codice C++
\lstset{
    language=C++,
    basicstyle=\ttfamily\small,
    keywordstyle=\color{blue},
    commentstyle=\color{gray},
    stringstyle=\color{orange},
    numbers=left,
    numberstyle=\tiny\color{gray},
    stepnumber=1,
    numbersep=10pt,
    tabsize=4,
    showspaces=false,
    showstringspaces=false,
    frame=single,
    breaklines=true,
    breakatwhitespace=true,
    captionpos=b
}

\title{Esercizio: Risoluzione di un'equazione di secondo grado}
\author{}
\date{}

\begin{document}

\maketitle
\thispagestyle{empty}
\pagestyle{empty}

\section*{Testo dell'esercizio}

Scrivi un programma in linguaggio \textbf{C++} che trovi le soluzioni reali dell'equazione di secondo grado nella forma:
\[
ax^2 + bx + c = 0
\]
dove i coefficienti \(a\), \(b\) e \(c\) sono numeri reali inseriti dall'utente.
Gestire i casi particolari in cui alcuni (o tutti i) coefficienti sono nulli.
Utilizzare una funzione per calcolare le soluzioni dell'equazione di secondo grado.
\section*{Esempi}

\begin{multicols}{2}
\noindent\textbf{Input:}
\begin{verbatim}
1 -3 2
\end{verbatim}
\noindent\textbf{Output:}
\begin{verbatim}
Due soluzioni reali distinte:
x1 = 2
x2 = 1
\end{verbatim}

\columnbreak

\noindent\textbf{Input:}
\begin{verbatim}
1 2 1
\end{verbatim}
\noindent\textbf{Output:}
\begin{verbatim}
Una soluzione reale doppia:
x = -1
\end{verbatim}
\end{multicols}

\vspace{0.5cm}

\begin{multicols}{2}
\noindent\textbf{Input:}
\begin{verbatim}
1 0 1
\end{verbatim}
\noindent\textbf{Output:}
\begin{verbatim}
Nessuna soluzione reale
(discriminante negativo).
\end{verbatim}

\columnbreak

\noindent\textbf{Input:}
\begin{verbatim}
0 2 -4
\end{verbatim}
\noindent\textbf{Output:}
\begin{verbatim}
Equazione di primo grado.
Soluzione: x = 2
\end{verbatim}
\end{multicols}

\vspace{0.5cm}

\begin{multicols}{2}
\noindent\textbf{Input:}
\begin{verbatim}
0 0 3
\end{verbatim}
\noindent\textbf{Output:}
\begin{verbatim}
L'equazione non ha soluzioni.
\end{verbatim}

\columnbreak

\noindent\textbf{Input:}
\begin{verbatim}
0 0 0
\end{verbatim}
\noindent\textbf{Output:}
\begin{verbatim}
L'equazione ha infinite soluzioni.
\end{verbatim}
\end{multicols}
\end{document}
