\documentclass[a4paper,12pt]{article}

\usepackage[utf8]{inputenc}
\usepackage[T1]{fontenc}
\usepackage{listings}
\usepackage{xcolor}
\usepackage{amsmath}
\usepackage{amssymb}
\usepackage{geometry}

% Stile per il codice C++
\lstset{
    language=C++,
    basicstyle=\ttfamily\small,
    keywordstyle=\color{blue},
    commentstyle=\color{gray},
    stringstyle=\color{orange},
    numbers=left,
    numberstyle=\tiny\color{gray},
    stepnumber=1,
    numbersep=10pt,
    tabsize=4,
    showspaces=false,
    showstringspaces=false,
    frame=single,
    breaklines=true,
    breakatwhitespace=true,
    captionpos=b
}

\title{Esercizio: Somma dei numeri da 0 a n (ricorsione)}
\author{}
\date{}

\begin{document}

\maketitle
\thispagestyle{empty}
\pagestyle{empty}

\section*{Testo dell'esercizio}

Scrivi un programma in linguaggio \textbf{C++} che:

\begin{itemize}
    \item chieda all'utente di inserire un numero intero positivo \( n \);
    \item calcoli, utilizzando una \textbf{funzione ricorsiva}, la somma dei numeri da \( 0 \) a \( n \), estremi inclusi;
    \item stampi a video il risultato ottenuto.
\end{itemize}

\section*{Esempio}

\noindent\textbf{Input:}
\begin{verbatim}
Inserisci un numero intero positivo: 5
\end{verbatim}

\noindent\textbf{Output:}
\begin{verbatim}
La somma dei numeri da 0 a 5 è: 15
\end{verbatim}



\end{document}

