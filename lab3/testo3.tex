\documentclass[a4paper,12pt]{article}

\usepackage[utf8]{inputenc}
\usepackage[T1]{fontenc}
\usepackage{listings}
\usepackage{xcolor}
\usepackage{geometry}

% Stile per il codice C++
\lstset{
    language=C++,
    basicstyle=\ttfamily\small,
    keywordstyle=\color{blue},
    commentstyle=\color{gray},
    stringstyle=\color{orange},
    numbers=left,
    numberstyle=\tiny\color{gray},
    stepnumber=1,
    numbersep=10pt,
    tabsize=4,
    showspaces=false,
    showstringspaces=false,
    frame=single,
    breaklines=true,
    breakatwhitespace=true,
    captionpos=b
}

\title{Esercizio: Puntatore al valore maggiore}
\author{}
\date{}

\begin{document}

\maketitle
\thispagestyle{empty}
\pagestyle{empty}

\section*{Testo dell'esercizio}

Scrivi un programma in linguaggio \textbf{C++} che:

\begin{itemize}
    \item legga da tastiera due numeri interi;
    \item crei un puntatore che punti al numero maggiore tra i due;
    \item visualizzi a schermo il valore del numero maggiore tramite il puntatore;
    \item modifichi, sempre attraverso il puntatore, il valore del numero maggiore portandolo a zero;
    \item mostri infine i valori aggiornati di entrambi i numeri.
\end{itemize}

\section*{Esempio}

\noindent\textbf{Input:}
\begin{verbatim}
5 10
\end{verbatim}

\noindent\textbf{Output:}
\begin{verbatim}
Il numero maggiore e': 10
Dopo l'azzeramento, i valori sono: a = 5, b = 0
\end{verbatim}

\end{document}

