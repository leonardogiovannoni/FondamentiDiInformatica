\documentclass[a4paper,12pt]{article}
\usepackage[utf8]{inputenc}
\usepackage[T1]{fontenc}
\usepackage{amsmath,amssymb}
\usepackage{enumitem}
\usepackage{geometry}
\usepackage{hyperref}
\geometry{margin=2.5cm}
\setlength{\parindent}{0pt}

\title{Esercizi di Programmazione (C++)}
\author{}
\date{}

\begin{document}
\maketitle
\setlist[itemize]{noitemsep, topsep=4pt}

\section{Triangolo rettangolo cavo di altezza $n$}
\textbf{Testo.} Leggi un intero $n \ge 1$ e stampa un \textbf{triangolo rettangolo cavo} di altezza $n$ usando gli asterischi \texttt{*}. Devono essere disegnati il cateto sinistro, l'ipotenusa e la base; gli spazi interni restano vuoti. Usa uno spazio doppio per gli spazi vuoti così da mantenere l'allineamento (ad esempio \texttt{"* "} e \texttt{"~~"}).
\newline 
\newline
\textbf{Esempio di esecuzione.}
\newline 
\newline
\noindent\textbf{Input:}
\begin{verbatim}
5
\end{verbatim}

\noindent\textbf{Output:}
\begin{verbatim}
* 
* *
*   *
*     *
* * * * *
\end{verbatim}

\section{Croce centrale in matrice $n \times n$ (con $n$ dispari)}
\textbf{Testo.} Leggi un intero $n \ge 1$ \textbf{dispari} e stampa una \textbf{croce} in una griglia $n \times n$ composta da asterischi \texttt{*}: una riga orizzontale e una colonna verticale passano entrambe per l'indice centrale $\lfloor n/2 \rfloor$. Le altre celle sono vuote (usa due spazi per mantenere l'allineamento). Se $n$ è pari o $n<1$, segnala un errore.
\newline 
\newline 
\textbf{Esempio di esecuzione.}
\newline 
\newline
\noindent\textbf{Input:}
\begin{verbatim}
5
\end{verbatim}

\noindent\textbf{Output:}
\begin{verbatim}
    *     
    *     
* * * * * 
    *     
    *   
\end{verbatim}
\section{Segno di $x$ e valore assoluto}
\textbf{Testo.} Leggi un intero $x$ e stampa due risultati:
\begin{itemize}
  \item $\mathrm{segno}(x)$, definito come $1$ se $x>0$, $0$ se $x=0$, $-1$ se $x<0$;
  \item $|x|$, il \textbf{valore assoluto} di $x$.
\end{itemize}

\textbf{Esempio di esecuzione.}


\noindent\textbf{Input:}
\begin{verbatim}
-17
\end{verbatim}

\noindent\textbf{Output:}
\begin{verbatim}
segno(x) = -1
|x| = 17
\end{verbatim}

\section{Primalità di $n$ e prossimo primo $\ge n$}
\textbf{Testo.} Leggi un intero $n$ e:
\begin{itemize}
  \item determina se $n$ è \textbf{primo} (stampa \texttt{si} oppure \texttt{no});
  \item stampa il \textbf{minimo numero primo} maggiore o uguale a $n$. Se $n \le 2$, il risultato è $2$.
\end{itemize}
L'algoritmo per il test di primalità può interrompersi a $\lfloor\sqrt{n}\rfloor$ controllando solo i divisori dispari dopo il 2.
\newline \newline
\textbf{Esempio di esecuzione.}
\newline 
\newline
\noindent\textbf{Input:}
\begin{verbatim}
30
\end{verbatim}

\noindent\textbf{Output:}
\begin{verbatim}
primo(n)? no
prossimo primo >= n: 31
\end{verbatim}

\section{Scambio di due interi e \textit{clamp} su un intervallo}
\textbf{Testo.} Leggi due interi $a$ e $b$, scambiali (\textit{swap}) e stampa i nuovi valori. Poi leggi tre interi $x$, $lo$, $hi$ con $lo \le hi$ e applica l'operazione di \textbf{clamp} a $x$:
\[
x \leftarrow 
\begin{cases}
lo & \text{se } x<lo\\
hi & \text{se } x>hi\\
x & \text{altrimenti}
\end{cases}
\]
Infine stampa il nuovo valore di $x$.
\newline \newline
\textbf{Esempio di esecuzione.}
\newline \newline

\noindent\textbf{Input:}
\begin{verbatim}
3 10
-2 0 7
\end{verbatim}

\noindent\textbf{Output:}
\begin{verbatim}
Dopo swap: a=10 b=3
Clamp(x): 0
\end{verbatim}

\end{document}

