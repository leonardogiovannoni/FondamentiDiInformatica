\documentclass[a4paper,12pt]{article}

\usepackage[utf8]{inputenc}
\usepackage[T1]{fontenc}
\usepackage{listings}
\usepackage{xcolor}
\usepackage{geometry}
\geometry{margin=2.5cm}

% Stile per il codice C++
\lstset{
    language=C++,
    basicstyle=\ttfamily\small,
    keywordstyle=\color{blue},
    commentstyle=\color{gray},
    stringstyle=\color{orange},
    numbers=left,
    numberstyle=\tiny\color{gray},
    stepnumber=1,
    numbersep=10pt,
    tabsize=4,
    showspaces=false,
    showstringspaces=false,
    frame=single,
    breaklines=true,
    breakatwhitespace=true,
    captionpos=b
}

\title{Esercizio: Quadrato cavo con diagonale principale}
\author{}
\date{}

\begin{document}

\maketitle
\thispagestyle{empty}
\pagestyle{empty}

\section*{Testo dell'esercizio}

Scrivi un programma in linguaggio \textbf{C++} che:

\begin{enumerate}
    \item \textbf{Legga} da tastiera un numero intero positivo \texttt{n}, rappresentante la dimensione del lato di un quadrato.
    \item \textbf{Disegni} un quadrato di lato \texttt{n} utilizzando il carattere \texttt{*}, in modo tale che:
    \begin{itemize}
        \item siano visibili i quattro bordi del quadrato;
        \item sia visibile anche la \textbf{diagonale principale} (da sinistra in alto a destra in basso);
        \item le altre posizioni interne siano vuote (spazi);
        \item ogni \texttt{*} sia intervallato da uno spazio.
    \end{itemize}
    \item Se l'utente inserisce un valore minore di 1, il programma deve visualizzare un messaggio di errore.
\end{enumerate}
Utilizzare una funzione per disegnare il quadrato.

\section*{Esempio}

\noindent\textbf{Input:}
\begin{verbatim}
5
\end{verbatim}

\noindent\textbf{Output:}
\begin{verbatim}
* * * * *
* *     *
*   *   *
*     * *
* * * * *
\end{verbatim}

\end{document}
