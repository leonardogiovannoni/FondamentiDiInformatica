\documentclass[a4paper,10pt]{article}

\usepackage[utf8]{inputenc}
\usepackage[T1]{fontenc}
% include math package
\usepackage{amsmath}
\usepackage{amssymb}
\usepackage{listings}
\usepackage{xcolor}
\usepackage{geometry}
\usepackage{multicol}
\usepackage{minted}

% diminuisci i margini
\geometry{margin=2cm}
\title{Esercizio: Funzione \texttt{clamp} con puntatori}
\author{}
\date{}

\begin{document}

\maketitle
\thispagestyle{empty}
\pagestyle{empty}

\section*{Testo dell'esercizio}

Scrivi un programma in linguaggio \textbf{C++} che:

\begin{itemize}
    \item legga da tastiera tre valori interi: un numero \( k \), un limite inferiore \( low \) e un limite superiore \( high \);
    \item definisca una funzione \texttt{clamp} che riceva come parametro un puntatore a intero e due limiti \( low \) e \( high \);
    \item faccia in modo che la funzione \texttt{clamp} modifichi il valore puntato, portandolo:
    \begin{itemize}
        \item a \( low \), se è minore di \( low \);
        \item a \( high \), se è maggiore di \( high \);
        \item lasciandolo invariato, se è già compreso tra \( low \) e \( high \).
    \end{itemize}
In termini matematici:
        \[
\text{clamp}\left(k\right) = 
\begin{cases}
    low, & \text{se } k < low \\[6pt]
    high, & \text{se } k > high \\[6pt]
    k, & \text{se } low \le k \le high
\end{cases}
\]
    % dove k è in z
    Dove \( k \in \mathbb{Z} \). da cui seguirà una funzione C++ del tipo:
\begin{minted}{cpp}
void clamp(int* pk, int low, int high) { .. }
\end{minted}
    \item stampi infine il valore di \( k \) aggiornato dopo la chiamata alla funzione.
\end{itemize}

\section*{Esempi}

\begin{multicols}{2}

\noindent\textbf{Input:}
\begin{verbatim}
15 0 10
\end{verbatim}
\noindent\textbf{Output:}
\begin{verbatim}
Il valore clamped è: 10
\end{verbatim}

\columnbreak

\noindent\textbf{Input:}
\begin{verbatim}
-5 0 10
\end{verbatim}
\noindent\textbf{Output:}
\begin{verbatim}
Il valore clamped è: 0
\end{verbatim}

\end{multicols}

\vspace{0.1cm}

\begin{multicols}{2}

\noindent\textbf{Input:}
\begin{verbatim}
7 0 10
\end{verbatim}
\noindent\textbf{Output:}
\begin{verbatim}
Il valore clamped è: 7
\end{verbatim}

\columnbreak

\noindent\textbf{Input:}
\begin{verbatim}
-15 -10 -5
\end{verbatim}
\noindent\textbf{Output:}
\begin{verbatim}
Il valore clamped è: -10
\end{verbatim}

\end{multicols}

\vspace{0.1cm}

\begin{multicols}{2}

\noindent\textbf{Input:}
\begin{verbatim}
0 0 10
\end{verbatim}
\noindent\textbf{Output:}
\begin{verbatim}
Il valore clamped è: 0
\end{verbatim}

\columnbreak

\noindent\textbf{Input:}
\begin{verbatim}
10 0 10
\end{verbatim}
\noindent\textbf{Output:}
\begin{verbatim}
Il valore clamped è: 10
\end{verbatim}

\end{multicols}

\end{document}

