\documentclass[a4paper,12pt]{article}

\usepackage[utf8]{inputenc}
\usepackage[T1]{fontenc}
\usepackage{listings}
\usepackage{xcolor}
\usepackage{geometry}
\geometry{margin=2.5cm}

\title{Esercizio 3: Calcolo del Massimo Comune Divisore (MCD)}
\author{}
\date{}

\begin{document}

\maketitle
\thispagestyle{empty}  % removes page number from the title page
\pagestyle{empty}      % removes page numbers from the rest of the document

\section*{Testo dell'esercizio}

Scrivi un programma in linguaggio \textbf{C++} che:

\begin{enumerate}
    \item \textbf{Legga} da tastiera due numeri interi positivi \texttt{a} e \texttt{b}.
    \item \textbf{Calcoli} il loro \textbf{massimo comune divisore (MCD)} utilizzando l'\textbf{algoritmo di Euclide}.
    \item \textbf{Stampi} a video il risultato.
\end{enumerate}

L’algoritmo di Euclide per il calcolo del MCD funziona nel seguente modo:

\begin{itemize}
    \item Finché \texttt{b} è diverso da 0:
    \begin{enumerate}
        \item si calcola \texttt{a \% b} (il resto della divisione di \texttt{a} per \texttt{b})
        \item si assegna alle variabili \texttt{a} e \texttt{b} rispettivamente i valori di \texttt{b} e del resto calcolato al passo precedente.
    \end{enumerate}
    \item Quando \texttt{b} diventa 0, il valore di \texttt{a} è il \textbf{massimo comune divisore}.
\end{itemize}

\section*{Esempio}

\noindent\textbf{Input:}
\begin{verbatim}
36 24
\end{verbatim}

\noindent\textbf{Output:}
\begin{verbatim}
Il massimo comune divisore è: 12
\end{verbatim}

\end{document}
