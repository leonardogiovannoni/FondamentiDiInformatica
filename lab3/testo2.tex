\documentclass[a4paper,12pt]{article}

\usepackage[utf8]{inputenc}
\usepackage[T1]{fontenc}
\usepackage{listings}
\usepackage{xcolor}
\usepackage{geometry}
\usepackage{minted}
\geometry{margin=2.5cm}

% Stile per il codice C++
\lstset{
    language=C++,
    basicstyle=\ttfamily\small,
    keywordstyle=\color{blue},
    commentstyle=\color{gray},
    stringstyle=\color{orange},
    numbers=left,
    numberstyle=\tiny\color{gray},
    stepnumber=1,
    numbersep=10pt,
    tabsize=4,
    showspaces=false,
    showstringspaces=false,
    frame=single,
    breaklines=true,
    breakatwhitespace=true,
    captionpos=b
}

\title{Esercizio: Passaggio per valore e per riferimento}
\author{}
\date{}

\begin{document}

\maketitle
\thispagestyle{empty}
\pagestyle{empty}

\section*{Testo dell'esercizio}

Scrivi un programma in linguaggio \textbf{C++} che permetta di comprendere la differenza tra:

\begin{itemize}
    \item il \textbf{passaggio per valore};
    \item il \textbf{passaggio per riferimento}.
\end{itemize}
Ogni funzione deve raddoppiare il valore del parametro ricevuto.

\noindent
Il programma deve eseguire questo main:

%\begin{enumerate}
%    \item Leggere da tastiera un numero intero \texttt{a}.
%    \item Stampare il valore iniziale di \texttt{a}.
%    \item Chiamare una funzione \texttt{f1(int x)} che riceve \texttt{a} per \emph{valore} e raddoppia il suo parametro. Dopo la chiamata, il valore di \texttt{a} nel \texttt{main} rimane invariato.
%    \item Chiamare una funzione \texttt{f2(int x)} che riceve \texttt{a} per \emph{valore}, ma ritorna il doppio del valore. Il risultato deve essere riassegnato ad \texttt{a} nel \texttt{main}.
%    \item Chiamare una funzione \texttt{f3(int \&x)} che riceve \texttt{a} per \emph{riferimento} e raddoppia direttamente la variabile originale.
%\end{enumerate}

\begin{minted}{cpp}

void f1(int x) {
    // Passaggio per valore
    // ..
}

int f2(int x) {
    // Passaggio per valore, con riassegnamento
    // ..
    return ...;
}

void f3(int &x) {
    // Passaggio per riferimento
    // ..
}

int main() {
    int a = 0;
    cout << "Inserisci un numero intero: ";
    cin >> a;
    cout << "\nValore iniziale: " << a << endl;
    f1(a); // Passaggio per valore
    cout << "Dopo f1(x): " << a << endl;
    a = f2(a); // Passaggio per valore, con riassegnamento
    cout << "Dopo x = f2(x): " << a << endl;
    f3(a); // Passaggio per riferimento
    cout << "Dopo f3(x): " << a << endl;

    return 0;
}
\end{minted}

%\noindent
%Il programma deve mostrare il valore della variabile \texttt{a} dopo ogni chiamata, per evidenziare le differenze tra i tre casi.

\section*{Esempio}

\noindent\textbf{Input:}
\begin{verbatim}
10
\end{verbatim}

\noindent\textbf{Output:}
\begin{verbatim}
Valore iniziale: 10
Dopo f1(x): 10
Dopo x = f2(x): 20
Dopo f3(x): 40
\end{verbatim}

\end{document}

