\documentclass[a4paper,12pt]{article}

\usepackage[utf8]{inputenc}
\usepackage[T1]{fontenc}
\usepackage{listings}
\usepackage{xcolor}
\usepackage{amsmath}
\usepackage{amssymb}
\usepackage{geometry}
\usepackage{minted}
\usepackage{multicol}
\geometry{margin=1.5cm}
% Stile per il codice C++
\lstset{
    language=C++,
    basicstyle=\ttfamily\small,
    keywordstyle=\color{blue},
    commentstyle=\color{gray},
    stringstyle=\color{orange},
    numbers=left,
    numberstyle=\tiny\color{gray},
    stepnumber=1,
    numbersep=10pt,
    tabsize=4,
    showspaces=false,
    showstringspaces=false,
    frame=single,
    breaklines=true,
    breakatwhitespace=true,
    captionpos=b
}

\title{Esercizio: Verifica ricorsiva di numeri pari in un array}
\author{}
\date{}

\begin{document}

\maketitle
\thispagestyle{empty}
\pagestyle{empty}

\section*{Testo dell'esercizio}

Scrivi un programma in linguaggio \textbf{C++} che:

\begin{itemize}
    \item legga da tastiera \( N = 10 \) numeri interi e li memorizzi nell'array;
    \item stampi a video il contenuto dell'array;
    \item utilizzi una \textbf{funzione ricorsiva} per verificare se tutti gli elementi dell'array sono numeri pari;
    \item stampi un messaggio che indichi se tutti gli elementi sono pari oppure no.
\end{itemize}
La funzione ricorsiva avrà una firma simile alla seguente:
\begin{minted}{cpp}
bool tutti_pari(int arr[], int n);
\end{minted}

\section*{Esempi}

\begin{multicols}{2}
\noindent\textbf{Input:}
\begin{minted}{text}
4 8 10
\end{minted}

\noindent\textbf{Output:}
\begin{minted}{text}
Array: 4 8 10 
Gli elementi sono tutti pari
\end{minted}

\columnbreak

\noindent\textbf{Input:}
\begin{minted}{text}
3 6 9
\end{minted}

\noindent\textbf{Output:}
\begin{minted}{text}
Array: 3 6 9 
Non tutti gli elementi sono pari
\end{minted}
\end{multicols}



%\end{multicols}

\end{document}

